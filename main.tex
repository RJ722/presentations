% Copyright 2004 by Till Tantau <tantau@users.sourceforge.net>.
%
% In principle, this file can be redistributed and/or modified under
% the terms of the GNU Public License, version 2.
%
% However, this file is supposed to be a template to be modified
% for your own needs. For this reason, if you use this file as a
% template and not specifically distribute it as part of a another
% package/program, I grant the extra permission to freely copy and
% modify this file as you see fit and even to delete this copyright
% notice. 

\documentclass{beamer}

\usetheme{Madrid}

\title{Removing Dead Code}
\author{Rahul Jha}
\date{PyCon India, 2017}
\subject{Removing dead code}


 \AtBeginSubsection[]{
    \begin{frame}<beamer>{Outline}
    \tableofcontents[currentsection,currentsubsection]
   \end{frame}
}
% import minted for code snippets
\usepackage{minted}

\begin{document}

\begin{frame}
  \titlepage
\end{frame}


\begin{frame}{About Me}
    Rahul Jha
    \begin{itemize}
        \item
        Electronics Engineering Student at ZHCET, AMU, Aligarh
        \item
        Website: \url{https://rj722.tech}
        \item
        Drop me a hello at \texttt{rahul722j@gmail.com}
    \end{itemize}
\end{frame}

\begin{frame}{Outline}
  \tableofcontents
\end{frame}

\section{Motivation}

\subsection{What is dead code?}

\begin{frame}{What is dead code?}{Definition}
    Dead code is the part of the source code of a program which can never be executed because there exists no control flow path to the code from the rest of the program.
    \newline
    \newline
    Furthermore, we can classify it into two types:
    \begin{itemize}
        \item {
            Unused code
        }
        \item {
            Unreachable code
        }
    \end{itemize}

\end{frame}

\begin{frame}{What is dead code?}{Unused code Examples}
    \begin{example}{Something is defined, but never used}
        \inputminted{python}{content/unused_code.py}
    \end{example}   
\end{frame}

\begin{frame}{What is dead code?}{Unreachable code Examples}
    \begin{example}{Code after return statements:}
        \inputminted{python}{content/code_after_return.py}
    \end{example}

    \begin{example}{Unsatisfiable Boolean conditions for if/else/while:}
        \inputminted{python}{content/unsatisfiable_conditions.py}
    \end{example}
\end{frame}

%% NOTES:
%% Vulture also detects unreachable code by looking for code after return, break, continue and raise statements
%% and by searching for unsatisfiable if- and while-conditions.

\subsection{How does dead code crawl in?}
\begin{frame}{How does dead code crawl in?}
    \begin{itemize}
        \item
            Refactoring
        \pause
        \item
            Programming errors
        \note{When we think a condition is satisfied, but it is not}
        \pause
        \item
            During debugging
        \pause
        \item
            Misspellings
    \end{itemize}    
\end{frame}

\subsection{Why remove dead code?}

\begin{frame}{Why remove dead code?}
  \begin{itemize}
  \item
    Sometimes, bugs are also introduced because of dead code
    \pause
  \item
    Ensures high level of code quality
    \note{Sometimes, people are confused as to why this function exists}
    \pause
  \item
    Shrinks program size
    \pause
  \item
    Maintenance overhead
  \end{itemize}
\end{frame}

\section{Vulture}

\subsection{What is vulture?}

\begin{frame}{Vulture}{An overview}
Repository: \url{https://github.com/jendrikseipp/vulture}

\begin{itemize}
    \item
    A command line tool developed by \href{https://github.com/jendrikseipp}{Jendrik Seipp}
        \pause
    \item
    Finds unused code in Python programs
        \pause
    \item
    If you run Vulture on both your library and test suite you can find untested code
        \pause
    \item
    Sometimes, reports false positives.
\end{itemize}
\end{frame}

\subsection{How to use vulture?}

\begin{frame}{How to use vulture?}{A quick tutorial}

\begin{itemize}
    \item{
        vulture can be installed using pip. It supports both Python 2 and 3.
    }
    \pause
    \item{
        For system wide installation, command:
        \begin{verbatim}

            pip install vulture

        \end{verbatim}
        \pause
    }
    \item{
        Run vulture
        \begin{verbatim}

            vulture myproject/

        \end{verbatim}
        \pause
    }
\end{itemize}

That's it :-)
\end{frame}

%% NOTES:
%% The provided arguments may be Python files or directories. For each directory Vulture analyzes all contained *.py files.
%% After you have found and deleted dead code, run Vulture again, because it may discover more dead code.

\begin{frame}{Additional arguments for vulture}{\texttt{--sort-by-size}}

\texttt{--sort-by-size} \begin{math} \implies \end{math} Sort the results by the number of lines of code.

\begin{example}
    \begin{verbatim}

        vulture myproject/ --sort-by-size

    \end{verbatim}

\end{example}
\end{frame}

%% NOTES:
%% This helps developers prioritize where to look for dead code first.

\begin{frame}{Additional arguments for vulture}{\texttt{--exclude}}
    
\texttt{--exclude PATTERN} \begin{math} \implies \end{math} Comma-separated list of paths to ignore 

\begin{itemize}
    \item
        PATTERNs can contain globbing
        characters (*, ?, [, ]).
    \item
        Treat PATTERNs without
        globbing characters as *PATTERN*.
\end{itemize}


\begin{example}
        \begin{verbatim}

        vulture . --exclude=*settings.py,docs/*.py

        \end{verbatim}
        
\end{example}
\end{frame}

\begin{frame}{Additional arguments for vulture}{\texttt{--min-confidence}}
    Every result has a confidence associated with it.
    \begin{itemize}
        \item
            You can use the \texttt{--min-confidence} flag to set the minimum confidence for code to be reported as unused.
        \item
            Use \textt{--min-confidence} 100 to only report code that is \alert{guaranteed} to be unused within the analyzed files.
    \end{itemize}
    
    \begin{example}
        \begin{verbatim}

            vulture myproject/ --min-confidence=50

        \end{verbatim}
    \end{example}
\end{frame}

\begin{frame}{Demo}
    \alert{Hooray! Let's do a demo now!}
    
\end{frame}

\subsection{False-positives?}

\begin{frame}{False-positives}{Why do they occur?}
    \begin{itemize}
        \item
            API endpoints are generally rendered as dead code, as they are user-oriented.
        \pause
        \item
            setup.py, config files, etc.
        \pause
        \item
            Code that is only called implicitly
    \end{itemize}
\end{frame}

\begin{frame}{Handling false-positives}{Whitelists}
    \begin{itemize}
        \item
            vulture provides an abstract \texttt{Whitelist} class
        \item
            Sub-class \alert{Whitelist} and create an attribute named exactly the same as the dead entity.
    \end{itemize}
\end{frame}

\begin{frame}{Example whitelist}
    \begin{block}{awesome\_lib.py}
        \inputminted{python}{content/yourAwesomeLib.py}
    \end{block}
    \begin{block}{whitelist\_awesome\_lib.py:}
        \inputminted{python}{content/whitelistYourAwesomeLib.py}
    \end{block}
\end{frame}

\begin{frame}{Marking unused variables}
    Vulture will ignore the variables if they start with an underscore
    e.g., in tuple assignments or function signatures.  \textt{(e.g., \_x, y = get\_pos())}.
\end{frame}

\subsection{How does vulture work?}

\begin{frame}{How does vulture work?}{Internals}
\begin{itemize}
    \item
        Vulture uses the \texttt{ast} module to build abstract syntax trees for all given files.
    \pause
    \item
        Constructs two different sets recording the names of defined and used objects.
    \pause
    \item
        Afterwards, it reports the objects which have been defined, but not used.
\end{itemize}
\end{frame}

\subsection*{vision}
\begin{frame}{What's going on in the vulture community right now?}
    \begin{itemize}
        \item
            We're currently extending whitelists for popular projects and frameworks, like django, flask, etc.
        \item
            Pull requests welcome! :-)
    \end{itemize}
\end{frame}

\section{Automated analysis through vulture}

\subsection{Adding vulture to CI tests}

\begin{frame}{Adding vulture to CI tests}
    vulture is a simple command line tool.
    Hence, a simple addition:

    \begin{verbatim}

        vulture project/ tests/ --exclude=project/*settings.py

    \end{verbatim}

    to the test script would do the work.

    \begin{example}
        \url{https://github.com/RJ722/example-vulture}
    \end{example}

\end{frame}

\subsection{Writing a script}

\begin{frame}{Custom script for dead code}{Easy to use vulture API}
    \begin{itemize}
        \item Useful when defining custom filters \pause or when defining custom output formats
        \pause
        \item \inputminted{python}{content/custom_script.py}
    \end{itemize}
\end{frame}

\begin{frame}{Writing a script}{Examples}
    \begin{example}
        \url{https://github.com/coala/coala-bears/blob/master/bears/python/VultureBear.py}
    \end{example}

\note{NOTE: We can also use this custom script in CI tests.}
\end{frame}



\subsection{VultureBear: An integration with coala}

\begin{frame}{coala}{Code Analysis Application}
    coala is a static code analysis application. It provides a unified interface for linting and fixing code.

    A \alert{bear} is is meant to do some analysis on source code. The source code will be provided by coala so the bear doesn’t have to care where it comes from or where it goes.

\end{frame}

%% Show coala flow chart diagram
%% A quick demo here - Run coala

\begin{frame}{VultureBear}
    Command:
    \begin{verbatim}

        coala -b VultureBear -f project/*.py

    \end{verbatim}
    
    and it's done.

    \newblock    
    \begin{example}
        \url{https://asciinema.org/a/82256}

    \end{example}

\end{frame}
%% Do a quick demo


\appendix
\section<presentation>*{\appendixname}
\subsection<presentation>*{For Further Reading}

\begin{frame}[allowframebreaks]
  \frametitle<presentation>{References}
    
  \begin{thebibliography}{10}
  \beamertemplatearticlebibitems

  \bibitem{Vulture Documentation}
    Vulture Documentation
    \newblock Much of the content for this presentation has been taken from official vulture documentation
    \newblock {\url{https://github.com/jendrikseipp/vulture}}

  \end{thebibliography}
\end{frame}

\begin{frame}{Conclusion}
    \begin{itemize}
        \item
            Install vulture using: \texttt{pip install vulture}
        \item
            Run using \texttt{vulture myproject/ --exclude=settings.py}
    \end{itemize}
\end{frame}

\end{document}
